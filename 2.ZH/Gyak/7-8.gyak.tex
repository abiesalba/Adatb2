\documentclass[a4paper,11.5pt, table]{article}

%%%%%% Basic packages begin %%%%%%
\usepackage[
	textwidth  = 160mm, 
	textheight = 230mm, 
	top        = 25mm, 
	bottom     = 30mm
]{geometry}
\usepackage[normalem]{ulem}
\usepackage[utf8]{inputenc}
\usepackage[T1]{fontenc}
\PassOptionsToPackage{defaults=hu-min}{magyar.ldf}
\usepackage[magyar]{babel}
%%%%%%% Basic packages end %%%%%%%

%%%%%% Packages required for this document begin %%%%%%
\usepackage{
	amsmath,   % Math mode
	amsthm,    % "note" environment
	amsfonts,  % "\mathbb{}" command
	paralist,  % "compactitem" and "compactenum" environment
	multirow,  % "\multirow{}{}" command
	float,     % "H" float specifier
	tikz,      % Basepackage for nearly all figures
	listings,  % Used for code snippets
	etaremune, % Reverse compactenum
%	graphicx   % For including images
	enumitem   %for alphabetical enumeration
}
\usepackage[unicode]{hyperref} % Clickable links
%%%%%%% Packages required for this document end %%%%%%%

%%%%%% TikZ options start %%%%%%
\usetikzlibrary{
	positioning, % Contains positioning utilities, such as "below = of" 
	calc,        % Adding coordinates
	math         % Needed for global variables
}

\tikzstyle{NodeBase} = [
	rectangle,
	text centered,
	draw = black
]

\definecolor{DefaultObjectColor}{gray}{0.9} % This is the default color of object in the TikZ pictures

\tikzstyle{arrow} = [
	thick,
	->,
	>=stealth
]
%%%%%%% TikZ options end %%%%%%%

%%%%%% lstlistings envvironment options start %%%%%%

\lstdefinestyle{customc}{ % C/C++ code snippet style
	belowcaptionskip = 1\baselineskip,
	breaklines       = true,
	frame            = L, % Double line on the left
	language         = C++,
	showstringspaces = false,
	basicstyle       = \ttfamily,
	keywordstyle     = \bfseries\color{green!40!black},
	stringstyle      = \color{orange},
	emphstyle        = \color{blue}, % Defined below
	tabsize          = 4,
	columns          = fullflexible,
}

\lstset{
	escapechar = @,
	style      = customc, % Default code snippet style
	%NOTE In order to use special characters in code snippets, one has to manually define them.
	literate   = {á}{{\'a}}1 {é}{{\'e}}1 {í}{{\'i}}1 {ó}{{\'o}}1 {ú}{{\'u}}1	{Á}{{\'A}}1 {É}{{\'E}}1 {Í}{{\'I}}1 {Ó}{{\'O}}1 {Ú}{{\'U}}1	{ö}{{\"o}}1 {ü}{{\"u}}1 {Ö}{{\"O}}1 {Ü}{{\"U}}1
	{ű}{{\H{u}}}1 {Ű}{{\H{U}}}1 {ő}{{\H{o}}}1 {Ő}{{\H{O}}}1
	{€}{{\euro}}1 {£}{{\pounds}}1 {~}{$\sim$}{1}	
}

%%%%%%% lstlistings envvironment options end %%%%%%%

%%%%%%%% Compilation error fix begin %%%%%%%%
\makeatletter
\expandafter\let\csname active@char\string?\endcsname\relax
\expandafter\let\csname active@char\string!\endcsname\relax
\expandafter\let\csname active@char\string:\endcsname\relax

\initiate@active@char{?}
\initiate@active@char{!}
\initiate@active@char{:}
\makeatletter
%%%%%%%%% Compilation error fix end %%%%%%%%%

\setlength{\parindent}{0mm}
\setlength{\parskip}{1em}
\setcounter{section}{0}

\begin{document}
	\begin{center}
		{\LARGE 7-8. gyak}
		\smallskip
		
		{\large 7. gyak}
	\end{center}
\begin{lstlisting}

/*****************************/

create table PLAN_TABLE (...); --expl.txt

DROP TABLE PLAN_TABLE;

EXPLAIN PLAN SET statement_id='ut1'  -- ut1 -> az utasításnak egyedi nevet adunk
FOR 
SELECT avg(fizetes) FROM nikovits.dolgozo;

SELECT * FROM PLAN_TABLE;

SELECT LPAD(' ', 2*(level-1))||operation||' + '||options||' + '||object_name terv
FROM plan_table
START WITH id = 0 AND statement_id = 'ut1'                 -- az utasítás neve szerepel itt
CONNECT BY PRIOR id = parent_id AND statement_id = 'ut1'   -- meg itt
ORDER SIBLINGS BY position;


SELECT SUBSTR(LPAD(' ', 2*(LEVEL-1))||operation||' + '||options||' + '||object_name, 1, 50) terv,
cost, cardinality, bytes, io_cost, cpu_cost
FROM plan_table
START WITH ID = 0 AND STATEMENT_ID = 'ut1'                 -- az utasítás neve szerepel itt
CONNECT BY PRIOR id = parent_id AND statement_id = 'ut1'   -- meg itt
ORDER SIBLINGS BY position;

select plan_table_output from table(dbms_xplan.display('plan_table','ut1','all'));
select plan_table_output from table(dbms_xplan.display());

CREATE TABLE dolgozo AS SELECT * FROM nikovits.dolgozo;
CREATE TABLE osztaly AS SELECT * FROM nikovits.osztaly;
CREATE TABLE Fiz_kategoria  AS SELECT * FROM nikovits.Fiz_kategoria;

EXPLAIN PLAN SET statement_id='ut2'  -- ut1 -> az utasításnak egyedi nevet adunk
FOR 
SELECT DISTINCT onev
FROM dolgozo NATURAL JOIN osztaly
WHERE fizetes > (
SELECT also
FROM FIZ_KATEGORIA
WHERE kategoria = 1
) AND fizetes <= (
SELECT felso
FROM FIZ_KATEGORIA
WHERE kategoria = 1
);

CREATE INDEX ind_onev
ON osztaly(onev);

select plan_table_output from table(dbms_xplan.display('plan_table','ut1','all'));
select plan_table_output from table(dbms_xplan.display('plan_table','ut2','all'));

---=== 1. feladat ===---
-- Adjuk meg úgy a lekérdezést, hogy egyik táblára se használjon indexet az oracle. 
EXPLAIN PLAN SET statement_id = 'cikk_orig'
FOR
SELECT sum(mennyiseg)
FROM nikovits.szallit NATURAL JOIN nikovits.cikk
WHERE szin = 'piros';

select plan_table_output from table(dbms_xplan.display('plan_table','cikk_orig','all'));

---=== 2. feladat ===---
-- Adjuk meg úgy a lekérdezést, hogy csak az egyik táblára használjon indexet az oracle. 
EXPLAIN PLAN SET statement_id = 'cikk_index_one_table'
FOR
SELECT /*+ index(c) */ sum(mennyiseg)
FROM nikovits.szallit NATURAL JOIN nikovits.cikk c
WHERE szin = 'piros';

select plan_table_output from table(dbms_xplan.display('plan_table','cikk_index_one_table','all'));

---=== 3. feladat ===---
-- Adjuk meg úgy a lekérdezést, hogy mindkét táblára használjon indexet az oracle.
EXPLAIN PLAN SET statement_id = 'cikk_index_two_tables'
FOR
SELECT /*+ index(c) index(sz) */ sum(mennyiseg)
FROM nikovits.szallit sz NATURAL JOIN nikovits.cikk c
WHERE szin = 'piros';

select plan_table_output from table(dbms_xplan.display('plan_table','cikk_index_two_tables','all'));

---=== 4. feladat ===---
-- Adjuk meg úgy a lekérdezést, hogy a két táblát SORT-MERGE módszerrel kapcsolja össze.

EXPLAIN PLAN SET statement_id = 'cikk_sort_merge'
FOR
SELECT /*+ USE_MERGE(sz c) */ sum(mennyiseg)
FROM nikovits.szallit sz NATURAL JOIN nikovits.cikk c
WHERE szin = 'piros';

select plan_table_output from table(dbms_xplan.display('plan_table','cikk_sort_merge','all'));

---===  5. feladat ===---
-- Adjuk meg úgy a lekérdezést, hogy a két táblát NESTED-LOOPS módszerrel kapcsolja össze. 

EXPLAIN PLAN SET statement_id = 'cikk_nested_loops'
FOR
SELECT /*+ USE_NL(sz c) */ sum(mennyiseg)
FROM nikovits.szallit sz NATURAL JOIN nikovits.cikk c
WHERE szin = 'piros';

select plan_table_output from table(dbms_xplan.display('plan_table','cikk_nested_loops','all'));

---=== 6. feladat ===---
-- Adjuk meg úgy a lekérdezést, hogy a két táblát HASH-JOIN módszerrel kapcsolja össze.

EXPLAIN PLAN SET statement_id = 'cikk_hash_join'
FOR
SELECT /*+ USE_HASH(sz c) */ sum(mennyiseg)
FROM nikovits.szallit sz NATURAL JOIN nikovits.cikk c
WHERE szin = 'piros';

select plan_table_output from table(dbms_xplan.display('plan_table','cikk_hash_join','all'));

---=== 7. feladat ===---
-- Adjuk meg úgy a lekérdezést, hogy a két táblát NESTED-LOOPS módszerrel kapcsolja össze,
-- és ne használjon indexet. 

EXPLAIN PLAN SET statement_id = 'cikk_nested_loops_no_index'
FOR
SELECT /*+ no_index(sz) no_index(c) USE_NL(sz c) */ sum(mennyiseg)
FROM nikovits.szallit sz NATURAL JOIN nikovits.cikk c
WHERE szin = 'piros';

select plan_table_output from table(dbms_xplan.display('plan_table','cikk_nested_loops_no_index','all'));

\end{lstlisting}

\begin{center}
	\smallskip
	
	{\large 8. gyak}
\end{center}

\begin{lstlisting}
---=== 1. feladat ===---
-- Adjuk meg azon szállítások összmennyiségét, ahol ckod=2 és szkod=2.
EXPLAIN PLAN SET statement_id = 'cikk2_orig'
FOR
SELECT sum(mennyiseg)
FROM nikovits.szallit sz NATURAL JOIN nikovits.cikk c
WHERE ckod = 2 AND szkod = 2;

select plan_table_output from table(dbms_xplan.display('plan_table','cikk2_orig','all'));

-- Adjuk meg úgy a lekérdezést, hogy ne használjon indexet.
EXPLAIN PLAN SET statement_id = 'cikk2_no_index'
FOR
SELECT /*+ no_index(sz) no_index(c) */ sum(mennyiseg)
FROM nikovits.szallit sz NATURAL JOIN nikovits.cikk c
WHERE ckod = 2 AND szkod = 2;

select plan_table_output from table(dbms_xplan.display('plan_table','cikk2_no_index','all'));

-- A végrehajtási tervben két indexet használjon, és képezze a sorazonosítók metszetét (AND-EQUAL).
-- nem jó!
EXPLAIN PLAN SET statement_id = 'cikk2_and_equal'
FOR
SELECT /*+ AND_EQUAL(sz) */ sum(mennyiseg)
FROM nikovits.szallit sz NATURAL JOIN nikovits.cikk c
WHERE ckod = 2 AND szkod = 2;

select plan_table_output from table(dbms_xplan.display('plan_table','cikk2_and_equal','all')); 

---=== 2. feladat ===---
-- Adjuk meg a Pecs-i telephely? szállítók által szállított piros cikkek összmennyiségét.

EXPLAIN PLAN SET statement_id = 'cikk3_orig'
FOR
SELECT sum(mennyiseg)
FROM nikovits.szallit NATURAL JOIN nikovits.cikk NATURAL JOIN nikovits.szallito
WHERE szin = 'piros' AND telephely = 'Pecs';

select plan_table_output from table(dbms_xplan.display('plan_table','cikk3_orig','all')); 

-- Adjuk meg úgy a lekérdezést, hogy a szallit táblát el?ször a cikk táblával join-olja az oracle.
EXPLAIN PLAN SET statement_id = 'cikk3_order_change'
FOR
SELECT /*+ leading(sz c) */ sum(mennyiseg)
FROM nikovits.szallit sz NATURAL JOIN nikovits.cikk c NATURAL JOIN nikovits.szallito szo
WHERE szin = 'piros' AND telephely = 'Pecs';

select plan_table_output from table(dbms_xplan.display('plan_table','cikk3_order_change','all')); 

-- Adjuk meg úgy a lekérdezést, hogy a szallit táblát el?ször a szallito táblával join-olja az oracle.
EXPLAIN PLAN SET statement_id = 'cikk3_order_change2'
FOR
SELECT /*+ leading(sz szo) */ sum(mennyiseg)
FROM nikovits.szallit sz NATURAL JOIN nikovits.cikk c NATURAL JOIN nikovits.szallito szo
WHERE szin = 'piros' AND telephely = 'Pecs';

select plan_table_output from table(dbms_xplan.display('plan_table','cikk3_order_change2','all')); 

---=== 3. feladat ===---
-- Adjuk meg azon szállítások összmennyiségét, ahol ckod=1 vagy szkod=2.
EXPLAIN PLAN SET statement_id = 'cikk4_orig'
FOR
SELECT sum(mennyiseg)
FROM nikovits.szallit sz NATURAL JOIN nikovits.cikk c
WHERE ckod = 1 OR szkod = 2;

select plan_table_output from table(dbms_xplan.display('plan_table','cikk4_orig','all')); 

--Adjuk meg úgy a lekérdezést, hogy ne használjon indexet.
EXPLAIN PLAN SET statement_id = 'cikk4_no_index'
FOR
SELECT /*+ no_index(sz) no_index(c) */ sum(mennyiseg)
FROM nikovits.szallit sz NATURAL JOIN nikovits.cikk c
WHERE ckod = 1 OR szkod = 2;

select plan_table_output from table(dbms_xplan.display('plan_table','cikk4_no_index','all'));

-- A végrehajtási tervben két indexet használjon, és képezze a kapott sorok unióját (CONCATENATION).
EXPLAIN PLAN SET statement_id = 'cikk4_concat'
FOR
SELECT /*+ index(sz) index(c) USE_CONCAT */ sum(mennyiseg)
FROM nikovits.szallit sz NATURAL JOIN nikovits.cikk c
WHERE ckod = 1 OR szkod = 2;

select plan_table_output from table(dbms_xplan.display('plan_table','cikk4_concat','all'));

\end{lstlisting}

\end{document}