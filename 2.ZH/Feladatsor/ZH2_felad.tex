\documentclass[a4paper,11.5pt, table]{article}

%%%%%% Basic packages begin %%%%%%
\usepackage[
	textwidth  = 160mm, 
	textheight = 230mm, 
	top        = 25mm, 
	bottom     = 30mm
]{geometry}
\usepackage[normalem]{ulem}
\usepackage[utf8]{inputenc}
\usepackage[T1]{fontenc}
\PassOptionsToPackage{defaults=hu-min}{magyar.ldf}
\usepackage[magyar]{babel}
%%%%%%% Basic packages end %%%%%%%

%%%%%% Packages required for this document begin %%%%%%
\usepackage{
	amsmath,   % Math mode
	amsthm,    % "note" environment
	amsfonts,  % "\mathbb{}" command
	paralist,  % "compactitem" and "compactenum" environment
	multirow,  % "\multirow{}{}" command
	float,     % "H" float specifier
	tikz,      % Basepackage for nearly all figures
	listings,  % Used for code snippets
	etaremune, % Reverse compactenum
%	graphicx   % For including images
	enumitem   %for alphabetical enumeration
}
\usepackage[unicode]{hyperref} % Clickable links
%%%%%%% Packages required for this document end %%%%%%%

%%%%%% TikZ options start %%%%%%
\usetikzlibrary{
	positioning, % Contains positioning utilities, such as "below = of" 
	calc,        % Adding coordinates
	math         % Needed for global variables
}

\tikzstyle{NodeBase} = [
	rectangle,
	text centered,
	draw = black
]

\definecolor{DefaultObjectColor}{gray}{0.9} % This is the default color of object in the TikZ pictures

\tikzstyle{arrow} = [
	thick,
	->,
	>=stealth
]
%%%%%%% TikZ options end %%%%%%%

%%%%%% lstlistings envvironment options start %%%%%%

\lstdefinestyle{customc}{ % C/C++ code snippet style
	belowcaptionskip = 1\baselineskip,
	breaklines       = true,
	frame            = L, % Double line on the left
	language         = C++,
	showstringspaces = false,
	basicstyle       = \ttfamily,
	keywordstyle     = \bfseries\color{green!40!black},
	stringstyle      = \color{orange},
	emphstyle        = \color{blue}, % Defined below
	tabsize          = 4,
	columns          = fullflexible,
}

\lstset{
	escapechar = @,
	style      = customc, % Default code snippet style
	%NOTE In order to use special characters in code snippets, one has to manually define them.
	literate   = {á}{{\'a}}1 {é}{{\'e}}1 {í}{{\'i}}1 {ó}{{\'o}}1 {ú}{{\'u}}1	{Á}{{\'A}}1 {É}{{\'E}}1 {Í}{{\'I}}1 {Ó}{{\'O}}1 {Ú}{{\'U}}1	{ö}{{\"o}}1 {ü}{{\"u}}1 {Ö}{{\"O}}1 {Ü}{{\"U}}1
	{ű}{{\H{u}}}1 {Ű}{{\H{U}}}1 {ő}{{\H{o}}}1 {Ő}{{\H{O}}}1
	{€}{{\euro}}1 {£}{{\pounds}}1 {~}{$\sim$}{1}	
}

%%%%%%% lstlistings envvironment options end %%%%%%%

%%%%%%%% Compilation error fix begin %%%%%%%%
\makeatletter
\expandafter\let\csname active@char\string?\endcsname\relax
\expandafter\let\csname active@char\string!\endcsname\relax
\expandafter\let\csname active@char\string:\endcsname\relax

\initiate@active@char{?}
\initiate@active@char{!}
\initiate@active@char{:}
\makeatletter
%%%%%%%%% Compilation error fix end %%%%%%%%%

\setlength{\parindent}{0mm}
\setlength{\parskip}{1em}
\setcounter{section}{0}

\begin{document}
	\begin{center}
		{\LARGE 2. ZH feladatsor kb. (B)}
		\smallskip
		
		{\large Min 33\% kell kb.}
	\end{center}
\section{Papíros (40p)}

	\begin{enumerate}
		\item Fogalmak
		\begin{enumerate}
			\item REDO ellenőrzőpont képzés lépései.
			\item ???
		\end{enumerate}
	
		\item Adatbázis helyreállítása napló alapján
		\begin{enumerate}
			\item Adott naplózásra(műveletek felsorolása) helyreállítás REDO alapján.
			\item Adott naplózásra (műveletek felsorolása) helyreállítás UNDO alapján.
		\end{enumerate}
	
		\item Megelőzési gráf felrajzolása adott ütemezésekre és az ütemezésekre konkurencia-ekvivalens összes soros ütemezés megadása.
		\begin{enumerate}
			\item Ütemezés: T1, T2, T3, T4 tranzakciókhoz tartozó műveletek sorozata (nem tranzakciók szerint sorba rendezve) pl. R1(A); R1(B); W3(B);...; W4(D)
			\item Előzőhöz hasonló, de azért persze nem pont ugyan olyan ütemezés.
		\end{enumerate}
		\item Még két fogalom az elmélet listából (2. részbeli)
		\begin{enumerate}
			\item Konzisztens-ekvivalens
			\item Példa holtpontra
		\end{enumerate}
	
	\end{enumerate}

\section{Gépes (25p)}
	
	\begin{enumerate}
		\item (7p) Adott 4 tábla: nikovits.cikk, szallit, szallito, projekt. Adjuk meg azoknak a piros cikkeknek számát, melyek pécsi projektben szerepeltek.
		\begin{enumerate}
			\item Úgy, hogy a joinok helyett NESTED LOOP legyen és egyik tábla se legyen indexelve.
			\item Úgy, hogy csak a cikk tábla legyen indexelve.
		\end{enumerate}
	
		\item (8p) SH felhasználó tábláira adjunk alábbi felépítésű tervezetet.
		
		\item (10p) SH felhasználó tábláira adjunk alábbi felépítésű tervezetet. (hasonló, mint a példa, ami a honlapon szerepel a 8-as gyakhoz.)
	\end{enumerate}

\end{document}