\documentclass[a4paper,11.5pt, table]{article}

%%%%%% Basic packages begin %%%%%%
\usepackage[
	textwidth  = 160mm, 
	textheight = 230mm, 
	top        = 25mm, 
	bottom     = 30mm
]{geometry}
\usepackage[normalem]{ulem}
\usepackage[utf8]{inputenc}
\usepackage[T1]{fontenc}
\PassOptionsToPackage{defaults=hu-min}{magyar.ldf}
\usepackage[magyar]{babel}
%%%%%%% Basic packages end %%%%%%%

%%%%%% Packages required for this document begin %%%%%%
\usepackage{
	amsmath,   % Math mode
	amsthm,    % "note" environment
	amsfonts,  % "\mathbb{}" command
	paralist,  % "compactitem" and "compactenum" environment
	multirow,  % "\multirow{}{}" command
	float,     % "H" float specifier
	tikz,      % Basepackage for nearly all figures
	listings,  % Used for code snippets
	etaremune, % Reverse compactenum
%	graphicx   % For including images
	enumitem   %for alphabetical enumeration
}
\usepackage[unicode]{hyperref} % Clickable links
%%%%%%% Packages required for this document end %%%%%%%

%%%%%% TikZ options start %%%%%%
\usetikzlibrary{
	positioning, % Contains positioning utilities, such as "below = of" 
	calc,        % Adding coordinates
	math         % Needed for global variables
}

\tikzstyle{NodeBase} = [
	rectangle,
	text centered,
	draw = black
]

\definecolor{DefaultObjectColor}{gray}{0.9} % This is the default color of object in the TikZ pictures

\tikzstyle{arrow} = [
	thick,
	->,
	>=stealth
]
%%%%%%% TikZ options end %%%%%%%

%%%%%% lstlistings envvironment options start %%%%%%

\lstdefinestyle{customc}{ % C/C++ code snippet style
	belowcaptionskip = 1\baselineskip,
	breaklines       = true,
	frame            = L, % Double line on the left
	language         = C++,
	showstringspaces = false,
	basicstyle       = \ttfamily,
	keywordstyle     = \bfseries\color{green!40!black},
	stringstyle      = \color{orange},
	emphstyle        = \color{blue}, % Defined below
	tabsize          = 4,
	columns          = fullflexible,
}

\lstset{
	escapechar = @,
	style      = customc, % Default code snippet style
	%NOTE In order to use special characters in code snippets, one has to manually define them.
	literate   = {á}{{\'a}}1 {é}{{\'e}}1 {í}{{\'i}}1 {ó}{{\'o}}1 {ú}{{\'u}}1	{Á}{{\'A}}1 {É}{{\'E}}1 {Í}{{\'I}}1 {Ó}{{\'O}}1 {Ú}{{\'U}}1	{ö}{{\"o}}1 {ü}{{\"u}}1 {Ö}{{\"O}}1 {Ü}{{\"U}}1
	{ű}{{\H{u}}}1 {Ű}{{\H{U}}}1 {ő}{{\H{o}}}1 {Ő}{{\H{O}}}1
	{€}{{\euro}}1 {£}{{\pounds}}1 {~}{$\sim$}{1}	
}

%%%%%%% lstlistings envvironment options end %%%%%%%

%%%%%%%% Compilation error fix begin %%%%%%%%
\makeatletter
\expandafter\let\csname active@char\string?\endcsname\relax
\expandafter\let\csname active@char\string!\endcsname\relax
\expandafter\let\csname active@char\string:\endcsname\relax

\initiate@active@char{?}
\initiate@active@char{!}
\initiate@active@char{:}
\makeatletter
%%%%%%%%% Compilation error fix end %%%%%%%%%

\setlength{\parindent}{0mm}
\setlength{\parskip}{1em}
\setcounter{section}{0}

\begin{document}
	\begin{center}
		{\LARGE Adatbázisok II}
		\smallskip
		
		{\large VI. Ellenőrzőkérdések}
	\end{center}
189. Milyen problémát kell megoldania a konkurrencia-vezérlésnek? (4 pont)
	\begin{compactitem}
		\item A tranzakciók közötti egymásra hatás az adatbázis-állapot inkonzisztenssé válását okozhatja, még akkor is, amikor a tranzakciók külön-külön megőrzik a konzisztenciát, és rendszerhiba sem történt. 
		\item \textit{MJ.: Akkor fordulhat elő ilyen, ha két, egyszerre futó tranzakció ugyan azt az adatrészt módosítja.}
	\end{compactitem}

190. Mit hívunk ütemezőnek? (2 pont)
	\begin{compactitem}
		\item Az adatbázis-kezelő azon részét hívjuk ütemezőnek (scheduler), amely a tranzakciós lépések szabályozásának feladatát végzi.
	\end{compactitem}

191. Mit hívunk ütemezésnek? (2 pont)
	\begin{compactitem}
		\item Az ütemezés \textit{(schedule)} egy vagy több tranzakció által végrehajtott lényeges műveletek időrendben vett sorozata, amelyben az egy tranzakcióhoz tartozó műveletek sorrendje megegyezik a tranzakcióban megadott sorrenddel. 
	\end{compactitem}

192. Milyen 2 módon biztosítja az ütemező a sorbarendezhetőséget? (2 pont)
	\begin{compactitem}
		\item Várakoztat, abortot rendel el, hogy a sorbarendezhetőséget biztosítsa.
	\end{compactitem}

193. Mit hívunk konfliktuspárnak? (2 pont)
	\begin{compactitem}
		\item A konfliktus \textit{(conflict)} vagy konfliktuspár olyan egymást követő műveletpár az ütemezésben, amelynek ha a sorrendjét felcseréljük, akkor legalább az egyik tranzakció viselkedése megváltozhat.
	\end{compactitem}

194. Milyen 3 esetben nem cserélhetjük fel a műveletek sorrendjét, mert inkonzisztenciát okozhatna? (3 pont)
\begin{compactitem}
	\item Legyen T$_{i}$ és T$_{j}$ két különböző tranzakció (i $\not=$ j).
	\begin{enumerate}[label= \alph*)]
		\item r$_{i}$(X); w$_{i}$(Y) konfliktus,
		\begin{compactitem}
			\item M
			ivel egyetlen tranzakción belül a műveletek sorrendje rögzített, és az adatbázis-kezelő ezt a sorrendet nem rendezheti át.
		\end{compactitem}
		
		\item w$_{i}$(X); w$_{j}$(X) konfliktus, 
		\begin{compactitem}
			\item Mivel mind a kettőt ugyan azt az adatot módosítja. (Ha felcserélnénk őket, más lehet az eredmény.)
		\end{compactitem}
		
		\item r$ _{i} $(X); w$ _{j} $(X) és w$ _{i} $(X); r$ _{j} $(X) is konfliktus. 
		\begin{compactitem}
			\item Ha megcserélnénk a sorrendet, egyrészt az írások miatt más lenne az olvasott adat, másrészt a felcserélt írások sorrendje miatt X értékének a 4 művelet utáni változata is megváltozhat. (És ez a csere azt is jelentené, hogy tranzakción belüli sorrendet cserélünk fel.)
		\end{compactitem}
	\end{enumerate}
\end{compactitem}

195. Mikor konfliktus-ekvivalens 2 ütemezés? (2 pont)
	\begin{compactitem}
		\item Azt mondjuk, hogy két ütemezés konfliktusekvivalens \textit{(conflict-equivalent)}, ha szomszédos műveletek nem konfliktusos cseréinek sorozatával az egyiket átalakíthatjuk a másikká. 
	\end{compactitem}

196. Mikor konfliktus-sorbarendezhető egy ütemezés? (2 pont)
	\begin{compactitem}
		\item Azt mondjuk, hogy egy ütemezés konfliktus-sorbarendezhető \textit{(conflict-serializable schedule)}, ha konfliktusekvivalens valamely soros ütemezéssel. 
		\item \textit{MJ.: Azaz nem konfliktusos cserékkel átvihető az egyik a másikba.}
	\end{compactitem}

\end{document}